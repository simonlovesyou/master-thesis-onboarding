\chapter{Discussion}
\label{chap:discussion}

The following chapter will first discuss the major validity of the framework in regards to the results in \ref{chap:result}. Secondly, we discuss limitations and drawbacks of the individual results in chapter \ref{chap:result} and their implication on the thesis.

\section{Validity}

The result of the thesis managed to prove quantitatively that the test participants expected ease of use and usefulness was exceeded after the use of the prototype, assuming that the recorded samples are from a normal distribution. From the qualitative result of the usability test, we can conclude that the test participants enjoyed the use of the prototype. With these results in mind, we also have to consider that
\begin{enumerate*}[label=(\(\arabic*\))]
  \item \label{enum:original-app}The conceptual app design, as designed without the framework, could have produced the same or similar results,
  \item \label{enum:framework-aspects}not all aspects of the framework was applied to the prototype,
  \item \label{enum:app-aspects}not all functionality and aspects of the reviewed app has been covered by the framework, and finally
  \item \label{enum:experience}another designer, with the aid of the framework, could have produced a different onboarding solution
\end{enumerate*}

For the case of \ref{enum:original-app},  testing the conceptual app design was not possible since the scope of the design did not include essential onboarding functionality (e.g. the user login, the creation of a grocery list etc). Time constraints were the main contributing factor to \ref{enum:framework-aspects} and \ref{enum:app-aspects}, that not all aspects of the framework were applied to the design. The fact that the framework is general and at times have a high abstraction level is the main contributor to \ref{enum:experience}.

\section{Limitations \& Drawbacks}
\subsection{Personas}
The personas designed was made to be as representative of the survey results as possible. The accurateness of the personas is limited by the results of the survey and the design of the survey, which is discussed in Section \ref{sec:discussion/survey} below.

\subsection{Survey}
\label{sec:discussion/survey}
A few of the questions in the survey could have been better formulated/designed. For instance, the options to question number six "I usually shop in bulk" was a Likert scale of 1---7 which indicated how much they agree with the statement (1: I disagree completely, 7: I agree completely). The survey results for this question can be quite hard to analyze, and the options should have rather consisted of likert style items to indicate how often they go to a store to shop in bulk, similar to question five.

The impact of this on the personas is minor since we believe we managed to get create representative personas none the less.

\subsection{Framework Remarks}
Analyzing the mobileapp app and its design with the framework could have yielded different results by another person since the framework is general and some aspects can have a high level of abstraction. Not all framework remarks was implemented in the prototype due to time constraints, for example, the proposed added functionality to recipes.

\subsection{Usability test}

Below are some general ways to improve the usability tests conducted:

\begin{description}
  \item[Tasks] The number of tasks identified could have been larger, but even if we would have identified more tasks we would not be able to test them all due to time constraints.
  \item[Scenarios] The number of scenarios could have been reduced by "merging" similar scenarios into one.
  \item[Measurements] The audio during the user tests could have been recorded. Even though the test supervisor managed to document the most critical events and comments of the test participant audio recording would have helped when reviewing the results of the scenarios.
\end{description}

\section{Future work}

\begin{description}
  \item[Validation] The framework can be more thoroughly validated through further user tests and its application on different types of mobile applications. A better validation method and study could also be applied to the study.
  \item[Theoretical Background] More theoretical models and research can be included in the theoretical background to further strengthen it, especially the acceleration and assimilation part of it.
  \item[Platform] The framework could be expanded to include tablets, which would introduce further research to the framework.
\end{description}

% WRITE IN PRESENT TENSE

% In the discussion, you write mor interpretatively and colorfully about the results. Whereas you kept it concise in the conclusion, you write more in-depth about the subject in the discussion section.
% Evaluate the research: you may discuss your expectations of possible causes of and consequences of the results, possible limitations and suggestions for follow-up research.

% Start your discussion with the validity of your research. Then discuss the results and indicate whether they meet your expectations.
% In this section, you will give explanations for meeting or not meeting these expectations.
% Describe how your results fit with the framework that you have drawn in the first chapter (introduction, motivation, theoretical framework, and research questions/hypotheses).
%Show how the finding provide new or different insights into what was already known.

% Present the limitation of your research in a new paragraph within the discussion. Describe which observations you can make based on the research results.
% If there are some side notes that can be made to the research or you were hindered by certain limitations, these issues can explain of the results you obtained.
% Name these, but also explain how there factors can be improved in the future research.

%% --------------------------- CHECKLIST -------------------------------------
% - The validity of the research is demonstrated.
% - New insights are explained.
% - The limitations of the research are discussed.
% - It is indicated whether expectations were justified.
% - Possible causes and consequences of the results are discussed.
% - Suggestions for possible follow-up research are made.
% - Own interpretations have been included in the discussion.
% - There are no suggestions for follow-up research that are too vague.
