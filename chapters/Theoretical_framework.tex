\chapter{Theoretical Framework}
\label{chap:theoretical_framework}
The theoretical framework is performed in two phases. The first phase \textit{Literature review} establish the current research environment in regards to onboarding, which lay ground for this study. The purpose of the second phase is to develop the \textit{theoretical background}, which major parts of the framework is based on. The theoretical background is then further supported by expert interviews and best practice evaluations of other mobile applications.

\begin{description}
  \item [Phase 1] The material is for the first phase is retrieved by searching the keywords "onboarding", "user onboarding" and "user sign up".
  \item [Phase 2] The material reviewed will lay grounds of what user experience (UX) is and what context it has to user onboarding. Some UX challenges and conventions related to mobile/smartphone devices is presented, and a vast coverage of general UX aspects such as "Flow", "Perception", "Memory", "Attention", "Emotion and Motivation" and "Learnability".
\end{description}

\section{Literature study}
% Literature reviews are designed to provide an overview of sources you have explored while researching a particular topic and **to demonstrate to your readers how your research fits into the larger field of study**
\subsection{User Onboarding}
There hasn't been a lot of academic research in the field of user onboarding, even though there's a lot of research regarding employee and manager onboarding (ref) (ref) (ref). The most notable contributions to the research field of user onboarding is by Oechsner \cite{Oechsner2016}, where she presents general guidelines for onboarding a user to a digital platform. The guidelines she presents are supported by psychology and IxD research, but fail to include the body of work of employee onboarding and the marketing aspect \textit{stickiness}. Also, her guidelines are meant to be general and may not alone aid the entire onboarding process of all platforms. Her study of user onboarding has been exhaustive, but with the circumstances related to the different usage model of mobile applications support the need for revisiting onboarding process.

\subsection{Onboarding}

The term "onboarding" is widely used in the context of organization socialization, and is described as the process where a new employee is introduced with the skills, knowledge and behaviors they need to succeed in their new organization \cite{Bauer2011}. In the field of management, recrution and human resource management, onboarding is an important aspect to reduce the time for the employee to be fully functional and engaged with the new company \cite{GraybillJolieO;HudsonCarpenterMariaTaesil;OffordJeromeJr;PiorunMary;Shaffer2013}.

\section{Theoretical Framework}
% User problems -> Mobile -> Value proposition

To better understand users and their behaviors we draw inspiration from cognitive and behavioral psychology. Cognitive psychology is a branch of psychology which study higher mental processes of the human mind such as attention, language use, memory, perception, problem solving, and thinking \footnote{\url{http://www.apa.org/research/action/glossary.aspx?tab=3}}. Behavioral psychology and its research field is focused on the environmental determinants of learning and behavior.

\subsection{Behavior}

\cite{Rasmussen1983} has identified three typical levels of performance. \textit{The skill-based behavior} represent the subconscious actions and activities which we perform on a "smooth, automated, and highly integrated patterns of behavior". This level of performance is based on a simple feedback loop, where a stimuli facilitate a motor output. \ignore{Provide example of this}

\textit{Rule-based behavior} are patterns of behavior which emerge from previous and similiar actions performed in a previous similiar occassion. The user acts in a goal-oriented manor where the user tries to achieve a goal, which is usually not explicitly stated, with the rules that has empirically evolved through previous successfull experiences. \ignore{Find evidence that rule and skill-based behavior might help the user with cognitive load}

The main differences of skill-based and rule-based behavior is that the person may not consciously be aware of the actions the user perform when perfomring actions on a skill-based level, and may not be able to recollect why such an action has been taken. The higher level rule-based actions are generally based on "explicit know-how, and the rules used can be reported by the person".

The third, and final level of performance as \cite{Rasmussen1983} has identified, is \textit{knowledge-based}. Knowledge-based performance is used during situations that the user is not familiar with, and where any know-how or rules cannot be used from previous experiences. \cite{Rasmussen1983} states that "In this situation, the goal is explicitly formulated, based on an anlysis of the environment and the overall aims of the person. Then a useful plan is developed-by selection-such that different plans are considered, and their effect tested against the goal, physically by trial and error, or conceptually by means of understand the functional properties of the environment and prediction of the effects of the plan is considered." At this level of abstraction, the user represent the system in an internal construct called \textit{mental model}.

%% Cognitive models -> Mental models ->

\subsection{Mental model}

A common concept in the field of cognitive psychology is the concept of \textit{mental models}. Mental models was first introduced by American philosopher Charles Sanders Peirce, where he argues that humans reason by a process which
"...examines the state of things
asserted in the premisses, forms a diagram of that state of things, perceives in the parts of that diagram relations not explicitly mentioned in the premisses, satisfies itself by mental experiments upon the diagram that these relations would always subsist, or at least would do so in a certain proportion of cases, and concludes their necessary, or probable, truth." \cite{Pierce1974}

This was further elaborated by the psychologist Kenneth Craik, where he proposes that humans construct "small-scale models" of external reality \cite{Craik1967}. These mental models enable us to use past events to be able to react to present events and anticipate future events, to reason, and to understand our environment. Since Craik's contributions, cognitive psychologist has argued that mental models are formed through current and general knowledge, perception and imagination \cite{Johnson-Laird2001} \ignore{Find some more citations to support this claim}. In the field of Human-Computer Interaction (HCI) mental models has sometimes been used interchangeably with cognitive and conceptual models, and their difference and usage might be confusing as \cite{Staggers1993} concludes. For the purpose of this paper, we'll be using Normans \cite{Norman2013a} definition of conceptual and mental model, where he states that mental model needs to be consistent with the designers conceptual model. Mental models are the models made from experience, instruction and training which users interact through \cite{Norman2013a}. Mental models of devices are created mostly through perceiving possible actions and its visible structures afforded by its interface \cite{Norman2013a}. Bla bla bla conceptual model mimic users mental model. 

If the conceptual model is not clearly communicated to the user and is not properly corrected through computer-human dialog, the user will have trouble performing the tasks they've set out to solve their problem. Studies of blame \ignore{Find these studies}, or \textit{attribution}, has shown that when a fault occurs in a system, the person in question is more likely to attribute the error to system than their own. Similarly, when fortune occurs to a person they are likely to attribute the fortune to their person and intelligence.  has

If the users mental model is not consistent with the conceptual model provided by the designer, the users mental model can be modified through a \textit{computer-human dialog}



The area of psychology that focuses on the environmental determinants of learning and behavior.

The theoretical framework will be categorized by the general journey of how smartphone users consume mobile apps, as identified by \cite{IGIGlobal2016}:

According to Foley et al., \cite{Foley1996} some of the primary goals of user-interface design is to increase the users \textit{speed of learning}, \textit{speed of use}, \textit{reduction of error rate}, encouragement of \textit{rapid recall} of how to use the interface and increase in \textit{attractiveness} to potentional users and buyers.

\begin{description}
  \item [Speed of learning] Speed of learning description
  \item [Speed of use] Speed of use description
  \item [Error rate]  Error rate description
  \item [Rapid recall] Rapid recall description
  \item [Attractiveness] Attractiveness description
\end{description}

\section{Mental models}

One of the best possible way to missing-word with the users mental model of the product is to do user research.

The communication between a user and a computer-based system is through a \textit{user-computer dialogue} \cite{Foley1996}, where the dialogue is communicated through a language of inputs and outputs. Similarly to regular human-to-human conversation, we may provide feedback if something is misunderstood or help the other person finish a sentence. The same is true for human-to-computer communication, where feedback is used to reinforce or discourage the users action, making the user adjust his or her mental model of the system.

Direct manipulation was first introduced by Shneiderman, which describe a system which inherit the following properties (p. 184 erryday things as well):

\begin{enumerate}
  \item Continuous representation of the object of interest
  \item Physical actions or labeled button presses instead of complex syntax
  \item Rapid incremental reversible operations whose impact on the object of interest is immediately visible
\end{enumerate}

Systems that inhibit these properties may have the

The gulf of execution and the gulf of evaluation.

\subsection{Mobile interaction models} %% Not the real name, but some similar things should exist
While modern smart phones are packed with a lot of functionality (GPS navigation, voice and data communication, multimedia consumption, gaming) their small form factor limit the possible input and outputs. The two primary means of input supported by these kind of devices are
\begin{enumerate}
  \item Touchscreen
  \item Sensors
\end{enumerate}
The sensors available are dependent on the device, but mostly encumbers accelorometers, gyroscopes and orientation sensors. The touchscreen enables the user to interact through two-dimensional surface gestures \cite{Ruiz2011}. It is also possible for the user to gesture in three dimensions using the sensors, but we will focus mostly on two-dimensional gestures.
The possible actions and models which we interact through with mobile devices are mostly based on hand-based gestures. The gesture model which we have for touchscreen-based mobile devices include tapping, swiping, shaking,

% ... skriv mer om mental models.

% A method used to gather knowledge that already exists in relation to a particular topic or problem.
% Conducting literature research provides insight into existing knowledge and theories related to your topic. It also ensures that your thesis has a string scientific grounding.
% Your goal is instead to critically discuss the most relevant ideas and information that you have found as part of your theoretical framework.

% The literature review serves as a real cornerstone for analyzing the problem being investigated. The basis for developing a comprehensive theoretical framework.

% Once you have a general idea of the problem and research questions you would like to address in your thesis, the first step is often to begin reviewing the literature.
% The insights into the existing knowledge and theories that you will gain through the literature review will also help you to establish a strong scientific starting point for the rest of your research.

%%%% LITERARURE ROADMAP
%% 1. Prepare
% The first step involves orienting yourself to the subject in order to get a more global picture of the area of inquiry. It also entails making a list of keywords, which will serve as the basis for the next step.

%% 2. Collect literature
% Does the same authors name keep popping up? This usually means this individual has done lot of research on the topic.

%% 3. Evaluate and select the literature
% Start with just the introduction and conclusion

%% 4. Process the literature
% Ask yourself the following questions while reading:
% - What is the problem being investigated and how is the research addressing it?
% - What are the key concepts and how are they being defined?
% - What theories and models does the author use?
% - What are the results and conclusions of the study?
% - How does this publication compare to related publications in this field?
% - How can I apply this research to my own research?
