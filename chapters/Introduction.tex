\chapter{Introduction}
\label{chap:introduction}

%% Establish the territory [the situation]
You only have one chance to make a good first impression. The amount of apps smartphone users download on a monthly basis has been steadily decreasing since the launch of the smartphone in 2004 (ref), and the founder of Mobilewalla suggest that users eventually uninstall 80-90\% of all downloaded apps\footnote{\url{http://usatoday30.usatoday.com/MONEY/usaedition/2012-01-31-App-Love-is-Fleeting\_ST\_U.htm}}. Similarly \cite{Perro2016} suggest that 75\% of apps get churned after 90 days, and 22\% of mobile users use an app only once before uninstalling\footnote{\url{http://www.slideshare.net/vaibhavkubadia75/mobile-web-vs-mobile-apps-27540693?from_search=1}}. Industry practitioners and researchers are increasingly mentioning the importance of app stickiness (retention) for a mobile applications success \cite{Perro2016} \cite{IGIGlobal2016} \cite{Kim2016}\footnote{\url{http://info.localytics.com/blog/an-open-letter-to-mary-meeker-we-are-in-a-mobile-engagement-crisis-localytics}}. And it is widely agreed across literature that stickiness will lead improved customer loyalty, positive word-of-mouth and revenue \cite{Reichheld2000} \cite{Srinivasan2002} \cite{Hsu2016a}.

The reason a user might uninstall an app and what makes an app "bad" has been previously researched \cite{Lin2012} \cite{Shklovski} \cite{Song2014}. The reasons found for uninstalling vary with privacy/security concerns, didn't meet users expectations, didn't use the app often etc. The term "app stickiness" is widely used in a marketing context for this type of problem to make sure that apps "sticks" to the users device. \cite{IGIGlobal2016} has linked user experience and its importance with app stickiness, and providing a good onboarding experience for the user will severely reduce abandonment rate\footnote{\url{https://clearbridgemobile.com/5-methods-for-increasing-app-engagement-user-retention/}} \footnote{\url{https://www.kahuna.com/blog/stop-spamming-your-users/}} and increase revenue \footnote{\url{http://www.appcues.com/blog/why-user-onboarding-is-the-most-important-part-of-the-customer-journey-by-2.6x/}}.

\ignore{Provide a definition of user onboarding}

In this paper, we present an analysis of psychology and user-experience on onboarding in mobile application. The ultimate aim is to provide guidelines for onboarding in different kinds of applications.

\section{Doberman}
Doberman is a digital design agency which help organizations and companies develop products and services that appeals customers. They have worked with companies like Sveriges Radio, Tele2 and Volvo Trucks\footnote{\url{http://doberman.co}}. They have been working with Sveriges Radio

\section{Aim}
The aim of this thesis is to, on a basis of earlier research of employee onboarding, gather a theoretical background of psychology and user-experience literature to explain/research/analyze factors which affect the process of user onboarding. We also aim to analyze different techniques used by mobile applications to onboard users to gather further insight into the industry practices.

\section{Objective}
The objective of this thesis is to develop a framework for user-experience designers and developers to aid them in the creation of onboarding users with mobile applications. The framework is meant to be general in regards to it being applicable to different kinds of mobile applications on different mobile platforms.

\section{Limitations}
While the objective of this thesis is to provide a general framework, some restrictions have been defined:
\begin{description}
  \item[Mobile phones] When discussing mobile phones in this thesis, we only consider smartphones which we define as cellular phones \footnote{A device which is primarily used by its user to access mobile network carrier through a network carrier (i.e. SIM or USIM card) services to make phone calls, send text messages etc.} "with advanced capabilities, which executes an identifiable operating system allowing users to extend its functionality with third party applications that are available from an application repository" \cite{Theoharidou2012}.
  \item[Tablets] Even though major parts of the framework may be applicable to apps designed and developed for tablets it is purely coincidental and not included in the objective for this thesis.
  \item[Aid] The framework is meant to be an aid to the user-experience designer and/or developer and further work and consideration might be required to provide a good user onboarding for a particular mobile app.
\end{description}

\section{Thesis outline}
\begin{enumerate*}
  \item As a starting point, we identify and establish an appropriate context with some key points and well-defined research problems with reference to existing literature (see Chapter \ref{chap:background}, below);
  \item A general theoretical background is developed of human behavior from psychology. The onboarding definition of Bradt et al. \cite{Bradt2009} was chosen as the foundation to build the background on (see Chapter \ref{chap:theoretical_framework}, below);
  \item several mobile apps are investigated to fit the theoretical model with the current practice of onboarding (see Chapter \ref{chap:theoretical_framework}, below);
  \item the best practice analysis and theoretical background is distilled into a framework for user onboarding (see Chapter \ref{chap:framework}, below)
  \item a user-testing session is conducted with numberOfParticipants of the mobile app mobileApp to test the proposed framework (see Chapter \ref{chap:result}, below);
  \item the results of the user testing is analyzed and discussed, and refinements to the framework is made (see Chapter \ref{Discussion}. below);
  \item finally, the study, its implications and future work is concluded (see Chapter \ref{Conclusion}, below); \ignore{https://www.researchgate.net/profile/Fred_Davis2/publication/35465050_A_Technology_Acceptance_Model_for_Empirically_Testing_New_End-User_Information_Systems/links/0c960519fbaddf3ba7000000/A-Technology-Acceptance-Model-for-Empirically-Testing-New-End-User-Information-Systems.pdf}
\end{enumerate*}
