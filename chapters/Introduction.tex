\chapter{Introduction}
\label{chap:introduction}

%% Establish the territory [the situation]
You only have one chance to make a good first impression. The amount of apps smartphone users download on a monthly basis has been steadily decreasing since the launch of the smartphone in 2004 (ref), and the founder of Mobilewalla suggest that users eventually uninstall 80-90\% of all downloaded apps \footnote{\url{http://usatoday30.usatoday.com/MONEY/usaedition/2012-01-31-App-Love-is-Fleeting\_ST\_U.htm}}. Similarly \cite{Perro2016} suggest that 75\% of apps get churned after 90 days, and 22\% of mobile users use an app only once before uninstalling \footnote{\url{http://www.slideshare.net/vaibhavkubadia75/mobile-web-vs-mobile-apps-27540693?from_search=1}}. Industry practictioners and researchers are increasingly mentioning the importance of app stickiness (retention) for a mobile applications success \cite{Perro2016} \cite{IGIGlobal2016} \cite{Kim2016} \footnote{\url{http://info.localytics.com/blog/an-open-letter-to-mary-meeker-we-are-in-a-mobile-engagement-crisis-localytics}}. And it's widely agreed across literature that stickiness will lead improved customer loyalty, positive word-of-mouth and revenue \cite{Reichheld2000} \cite{Srinivasan2002} \cite{Hsu2016a}. \cite{IGIGlobal2016} has linked user experience (UX) and its importance with app stickiness.

The reason a user might uninstall an app and what makes an app "bad" has been previously researched \cite{Lin2012} \cite{Shklovski} \cite{Song2014}. \cite{IGIGlobal2016} has  The reasons found for uninstalling vary with privacy/security concerns, didn't meet users expectations, didn't use the app often etc. The term "app stickiness" is a term widely used in a marketing context for this type of problem,

Some of the reasons for uninstalling might be hard to account for, but designers and developers can at least help the user understand what the apps' value proposition is, and if the value proposition meets the users' need and expectation, help the user transition from a novice to a proficient user.

The term onboarding is a quite recent term \cite{Dai2007}. The term onboarding stems from a business, management and human-resource context, and it describes a process to introduce and help a new employee become an effective member of their new organization by teaching them the knowledge, skills and behaviors required to succeed \cite{Bauer2011}. Some study has identified that onboarding in a business context has a lot in common with user onboarding, since it contains a lot of the same aspects of bringing them up to speed and making them comfortable in their new environment. There are a lot of different opinions and thoughts about what a onboarding process entail, and when a employee or user has been successfully "onboarded".

Successfully onboarding a user will increase user retention and may increase revenue \footnote{\url{http://www.appcues.com/blog/why-user-onboarding-is-the-most-important-part-of-the-customer-journey-by-2.6x/}}, so its importance in application and mobile development is apparent.

\section{Terminology}
\begin{description}
  \item [User-experience] User-experience (UX) "...is about technology that fulfills more than just instrumental needs in a way that acknowledges its use as a subjective, situated, complex and dynamic encounter. UX is a consequence of a user’s internal state (predispositions, expectations, needs, motivation, mood, etc.), the characteristics of the designed system (e.g. complexity, purpose, usability, functionality, etc.) and the context (or the environment) within which the interaction occurs (e.g. organizational/social setting, meaningfulness of the activity, voluntariness of use, etc.)" \cite{Hassenzahl2006}
\end{description}

\section{Doberman}
Doberman is a digital design agency which help organizations and companies develop products and services that appeals customers. They have worked with companies like Sveriges Radio, Tele2 and Volvo Trucks. More information about Doberman can be found at their website\footnote{\url{http://doberman.co}}.

\section{Objective}
The study is set to develop a framework for user onboarding for usage with mobile apps, which is evaluated through user testing on mobile applications. The framework is meant to be general in regards to mobile applications, and its application support different kinds of mobile apps on different platforms.

\section{Limitations}
Any suggestions?

\section{Goals}
The goal is to produce a set of design guidelines which help the designer provide a good onboarding experience for a user in a mobile application.
