\chapter{Introduction}
Since the launch of the smart phone 1000 million of mobile apps has been downloaded. The amount of apps users download on a monthly basis has been steadily decreasing, and a large portion of the apps we download are temporary. The mobile market is switching from user acqusition to user retention

Since the introduction of the smart phone and mobile app 1000 million apps has been downloaded, bla bla market is switching from user acqusition to user retention. One important aspect of user retention is to limit the friction of the users first-time use

% WRITE IN PRESENT TENSE
% C.A.R.S model

% Describe the topic of your thesis
% Formulate the problem statement
% Write an overview of your thesis

% Although the introduction is at the beginning of your thesis, this placement does not mean that you must finish the introduction before you can start the rest of your research. The further you get in your research, the easier it will be to write a good introduction that is to the point. Thus, it is no disaster if you ca not write a perfect introduction right away.

%% 	PURPOSE OF THE INTRODUCTION
% - Introduce the topic. What is the purpose of the study and what is the topic?
% - Gain the readers interest. Make sure that you get the readers attention by using clear examples from recent news items or everyday life.
% - Demonstrate the relevance of the study. Convince the reader of the scientific and practical relevance.

%%%%%% PARTS OF THE INTRODUCTION
%%% - Motivation
% What is the motive for your research? This can be a recent news item or something that has always interested you.

%%% - Scope
% Based on the motivation or problem indication, you describe the topic of your thesis. Make sure that you directly define the topic of your research.

%%% - Theoretical and practical relevance of the research
% Using arguments, state the scientific relevance of your research.
% Highlight here the discussion chapters of studies that you are going to use for your own research.
% Explain the practical use of your research.
% When you are writing a thesis for a company, you will find that the scientific relevance is much more difficult to demonstrate. On the other hand, it should be easier to show the practical benefit. Think of, for example, the practical applications for the entire industry.

%%% - Current scientific situation
% Specify the most important scientific articles that relate to your topic and you briefly explain them.
% Show that many studies have been conducted around the topic, and that you won't get stuck due to finding too little information on your topic.

%%% - Objective of the study and the problem statement
% Describe the objective of your study and the problem statement that you have formulated.

%%% - Brief description of the research design
% Provide a brief summary of your research design. How, where, when and with whom are you going to conduct your research?

%%% - Thesis outline
% Briefly describe how your thesis is constructed. Summarize each chapter briefly in one paragraph at the most, but preferably in one sentence.
% Make sure your thesis outline is not repetitively phrased because it does not vary its word choice.

% Do not repeat yourself and only write down what is important to introduce your topic and research.

%% --------------------------- CHECKLIST -------------------------------------
% - Introduction of the research is written with a stimulating topic.
% - The topic is limited
% - The scientific relevance is demonstrated (not always applicable)
% - The practical relevance is demonstrated
% - The most important scientific articles about the topic are summarized
% - The objective is formulated
% - The problem statement is formulated
% - The conceptual framework is determined
% - The research question or hypotheses are formulated
% - The research design is described briefly
% - The thesis overview is added

\section{}

\section{Objective}

\section{Limitations}

\section{Goals}

\begin{itemize}
	\item
	\item
	\item
\end{itemize}

\section{Terminology}

\begin{tabular}{ | p{5.5cm} | p{6.5cm} |  }
	\hline
     	\multicolumn{2}{|c|}{Common terms and their explanation} \\
     	\hline
	     Field of View (FOV) & Explanation goes here. \\
    	\hline
     	Head-Mounted Display (HMD) & Explanation goes here. \\
    	\hline
     	Virtual Reality (VR) & Explanation goes here. \\
    	\hline
	Virtual Environment (VE) & Explanation goes here. \\
     	\hline
    	Omidirectional video (ODV) & Explanation goes here. \\
    	\hline
     	360� video & Explanation goes here. \\
     	\hline
     	3D video & Explanation goes here. \\
     	\hline
     	Hypervideo & Explanation goes here. \\
	\hline
\end{tabular}
