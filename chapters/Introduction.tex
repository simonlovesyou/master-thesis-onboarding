\chapter{Introduction}
\label{chap:introduction}

%% Establish the territory [the situation]
You only have one chance to make a good first impression. The amount of apps smartphone users download on a monthly basis has been steadily decreasing since the launch of the smartphone in 2004 [1], and the founder of Mobilewalla suggest that users eventually uninstall 80-90\% of all downloaded apps \footnote{\url{http://usatoday30.usatoday.com/MONEY/usaedition/2012-01-31-App-Love-is-Fleeting\_ST\_U.htm}}, and 22\% of mobile users use an app only one before uninstall \footnote{\url{http://www.slideshare.net/vaibhavkubadia75/mobile-web-vs-mobile-apps-27540693?from_search=1}}. The reason a user might uninstall an app and what makes an app "bad" has been previously researched \cite{Lin2012} \cite{Shklovski} \cite{Song2014}. The reasons found for uninstalling vary with privacy/security concerns, didn't meet users expectations, didn't use the app often etc. Some of the reasons for uninstalling might be hard to account for, but designers and developers can at least help the user understand what the apps' value proposition is, and if the value proposition meets the users' need and expectation, help the user transition from a novice to a proficient user.

The term onboarding is a quite recent term \cite{Dai2007}. The term onboarding stems from a business, management and human-resource context, and it describes a process to introduce and help a new employee become an effective member of their new organization by teaching them the knowledge, skills and behaviors required to succeed \cite{Bauer2011}. Some study has identified that onboarding in a business context has a lot in common with user onboarding, since it contains a lot of the same aspects of bringing them up to speed and making them comfortable in their new environment. There are a lot of different opinions and thoughts about what a onboarding process entail, and when a employee or user has been successfully "onboarded".

Successfully onboarding a user will increase user retention (ref), and

There hasn't been a lot of academic research in the field of user onboarding, even though there's a lot of research regarding employee and manager onboarding (ref) (ref) (ref). The most notable contributions to the research field of user onboarding is by Oechsner, where she presents general guidelines for onboarding a user to a digital platform. Her guidelines are supported by psychology and IxD research, but fail to include the body of work of employee onboarding. Also, her guidelines are meant to be general and may not alone aid the the entire process of all platforms. This

% Define onboarding in a business context
% Talk about the problems inherited, and explain it in an user onboarding context.

, as defined by  is a process designed by developers and designers to help the user familiarize themselves with the application, its value proposition and. The onboarding process helps the user grow from a noob to a re

% WRITE IN PRESENT TENSE
% C.A.R.S model

% Describe the topic of your thesis
% Formulate the problem statement
% Write an overview of your thesis

% Although the introduction is at the beginning of your thesis, this placement does not mean that you must finish the introduction before you can start the rest of your research. The further you get in your research, the easier it will be to write a good introduction that is to the point. Thus, it is no disaster if you ca not write a perfect introduction right away.

%% 	PURPOSE OF THE INTRODUCTION
% - Introduce the topic. What is the purpose of the study and what is the topic?
% - Gain the readers interest. Make sure that you get the readers attention by using clear examples from recent news items or everyday life.
% - Demonstrate the relevance of the study. Convince the reader of the scientific and practical relevance.

%%%%%% PARTS OF THE INTRODUCTION
%%% - Motivation
% What is the motive for your research? This can be a recent news item or something that has always interested you.

%%% - Scope
% Based on the motivation or problem indication, you describe the topic of your thesis. Make sure that you directly define the topic of your research.

%%% - Theoretical and practical relevance of the research
% Using arguments, state the scientific relevance of your research.
% Highlight here the discussion chapters of studies that you are going to use for your own research.
% Explain the practical use of your research.
% When you are writing a thesis for a company, you will find that the scientific relevance is much more difficult to demonstrate. On the other hand, it should be easier to show the practical benefit. Think of, for example, the practical applications for the entire industry.

%%% - Current scientific situation
% Specify the most important scientific articles that relate to your topic and you briefly explain them.
% Show that many studies have been conducted around the topic, and that you won't get stuck due to finding too little information on your topic.

%%% - Objective of the study and the problem statement
% Describe the objective of your study and the problem statement that you have formulated.

%%% - Brief description of the research design
% Provide a brief summary of your research design. How, where, when and with whom are you going to conduct your research?

%%% - Thesis outline
% Briefly describe how your thesis is constructed. Summarize each chapter briefly in one paragraph at the most, but preferably in one sentence.
% Make sure your thesis outline is not repetitively phrased because it does not vary its word choice.

% Do not repeat yourself and only write down what is important to introduce your topic and research.

%% --------------------------- CHECKLIST -------------------------------------
% - Introduction of the research is written with a stimulating topic.
% - The topic is limited
% - The scientific relevance is demonstrated (not always applicable)
% - The practical relevance is demonstrated
% - The most important scientific articles about the topic are summarized
% - The objective is formulated
% - The problem statement is formulated
% - The conceptual framework is determined
% - The research question or hypotheses are formulated
% - The research design is described briefly
% - The thesis overview is added

\section{}

\section{Objective}

\section{Limitations}

\section{Goals}

\begin{itemize}
	\item
	\item
	\item
\end{itemize}

\section{Terminology}

\begin{tabular}{ | p{5.5cm} | p{6.5cm} |  }
	\hline
     	\multicolumn{2}{|c|}{Common terms and their explanation} \\
     	\hline
	     Field of View (FOV) & Explanation goes here. \\
    	\hline
     	Head-Mounted Display (HMD) & Explanation goes here. \\
    	\hline
     	Virtual Reality (VR) & Explanation goes here. \\
    	\hline
	Virtual Environment (VE) & Explanation goes here. \\
     	\hline
    	Omidirectional video (ODV) & Explanation goes here. \\
    	\hline
     	360� video & Explanation goes here. \\
     	\hline
     	3D video & Explanation goes here. \\
     	\hline
     	Hypervideo & Explanation goes here. \\
	\hline
\end{tabular}
