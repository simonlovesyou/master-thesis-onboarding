\chapter{Study of User Onboarding}

% A method used to gather knowledge that already exists in relation to a particular topic or problem.
% Conducting literature research provides insight into existing knowledge and theories related to your topic. It also ensures that your thesis has a string scientific grounding.
% Your goal is instead to critically discuss the most relevant ideas and information that you have found as part of your theoretical framework.

% The literature review serves as a real cornerstone for analyzing the problem being investigated. The basis for developing a comprehensive theoretical framework.

% Once you have a general idea of the problem and research questions you would like to address in your thesis, the first step is often to begin reviewing the literature.
% The insights into the existing knowledge and theories that you will gain through the literature review will also help you to establish a strong scientific starting point for the rest of your research.

%%%% LITERARURE ROADMAP
%% 1. Prepare
% The first step involves orienting yourself to the subject in order to get a more global picture of the area of inquiry. It also entails making a list of keywords, which will serve as the basis for the next step.

%% 2. Collect literature
% Does the same authors name keep popping up? This usually means this individual has done lot of research on the topic.

%% 3. Evaluate and select the literature
% Start with just the introduction and conclusion

%% 4. Process the literature
% Ask yourself the following questions while reading:
% - What is the problem being investigated and how is the research addressing it?
% - What are the key concepts and how are they being defined?
% - What theories and models does the author use?
% - What are the results and conclusions of the study?
% - How does this publication compare to related publications in this field?
% - How can I apply this research to my own research?
