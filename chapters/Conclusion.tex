\chapter{Conclusion}
\label{chap:conclusion}

A general design framework for onboarding users onto a mobile app was developed based on a large study of behavioral psychology, interaction design research, cognitive psychology and best practice analysis of common onboarding techniques. As the framework was applied to the current design concept of ICA Handla into an interactive prototype their expected perceived ease of use and perceived usefulness of the app exceeded their expectation as they used the prototype. The subjective measures also indicated that the test participants enjoyed using the interactive prototype with minor user errors. As a result, the framework can be considered to be a useful tool when designing the onboarding of users onto a mobile app. However, a larger number of different apps are required to be tested to validate its use with different types of apps since only one type of mobile app has been tested.

During the thesis we managed to pinpoint and a lot of aspect that affect the onboarding experience for users which we present in a general design framework. The most prominent conclusion is that successfully onboarding a user is \textit{hard}, and a lot has to be considered when designing the onboarding process.

% PRESENT TENSE
% Should be between 200-400 words.

% Answer your research question. Reiterate the research question, but integrate an explanation of it into the rest of the sections discussion.

%% --------------------------- CHECKLIST -------------------------------------
% - The research questions have been answered.
% - The main question or problem statement has been answered.
% - The hypotheses have been confirmed or refused.
% - The right verb tense has been used.
% - No issues are interpreted.
% - No new information has been given.
% - No examples are used.
% - No extraneous information is provided.
% - No passages from the results have been cut and pasted.
% - The first person has not been used.
