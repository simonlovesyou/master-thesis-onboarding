\chapter{Background}
\label{chap:background}
This section introduce and explain key points and concepts which are essential to understand the context and application of this study.

\section{User Experience}
User experience design is a conceptual design practice, which has its roots in human-computer interaction, functional design and ergonomics. Human-computer interaction is alalskdfma, The term \textit{user experience design} was first coined by Norman, a user experience architect, who thought that the terms human interface and usability were too narrow \cite{Merholz2008}. Norman wanted to cover all different aspects of the user's experience with a product "including industrial design graphics, the interface, the physical interaction and the manual". Since the fields conception and as it has been used in scientific literature and professionaly, there's been exhaustive literary and scientific efforts to define the field and its concerns \cite{Law2008} \cite{Law2009} \cite{Forlizzi2000}. Hassenzahl \cite{Hassenzahl2006}, a german psychologist, has defined it as "...user experience is about technology that fulfills more than just instrumental needs in a way that acknowledges its use as a subjective, situated, complex and dynamic encounter. UX is a consequence of a user’s internal state (predispositions, expectations, needs, motivation, mood, etc.), the characteristics of the designed system (e.g. complexity, purpose, usability, functionality, etc.) and the context (or the environment) within which the interaction occurs (e.g. organizational/social setting, meaningfulness of the activity, voluntariness of use, etc.)".

\subsection{Mobile UX}

\section{Onboarding}
The term "onboarding" is widely used in the context of organization socialization, and is described as the process where a new employee is introduced with the skills, knowledge and behaviors they need to succeed in their new organization \cite{Bauer2011}. In the field of management, recruiting and human resource management, onboarding is an important aspect to reduce the time for the employee to be fully functional and engaged with the new company \cite{GraybillJolieO;HudsonCarpenterMariaTaesil;OffordJeromeJr;PiorunMary;Shaffer2013}.

\subsection{User onboarding}
