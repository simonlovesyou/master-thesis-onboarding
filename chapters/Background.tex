\chapter{Background}
\label{chap:background}
This section introduce and explain key points and concepts which are essential to understand the context and application of this study.

\section{User Experience}
User experience design is a conceptual design practice, which has its roots in human-computer interaction, functional design and ergonomics. Human-computer interaction is alalskdfma, The term \textit{user experience design} was first coined by Norman, a user experience architect, who thought that the terms human interface and usability were too narrow \cite{Merholz2008}. Norman wanted to cover all different aspects of the users experience with a product "including industrial design graphics, the interface, the physical interaction and the manual". Since the fields conception and as it has been used in scientific literature and professionally, there is been exhaustive literary and scientific efforts to define the field and its concerns \cite{Law2008} \cite{Law2009} \cite{Forlizzi2000}. Hassenzahl \cite{Hassenzahl2006}, a German psychologist, has defined it as "...user experience is about technology that fulfills more than just instrumental needs in a way that acknowledges its use as a subjective, situated, complex and dynamic encounter. UX is a consequence of a user’s internal state (predispositions, expectations, needs, motivation, mood, etc.), the characteristics of the designed system (e.g. complexity, purpose, usability, functionality, etc.) and the context (or the environment) within which the interaction occurs (e.g. organizational/social setting, meaningfulness of the activity, voluntariness of use, etc.)".

\subsection{Mobile UX}
With the pattern of discontinued use, "micro-breaks" \cite{McGregor2014a}, the small form factor and the wide range of different possible contexts of use \cite{Wang2016}, designing interaction for mobile devices requires its own research methods, tools and infrastructure to overcome these challenges \cite{Nakhimovsky2009}. The possibilities of mobile devices enable us to communicate more with our family and friends, stay more informed, innovative, productive etc \cite{Wang2016}.
\section{Onboarding}
The term onboarding is a quite recent term \cite{Dai2007}. Onboarding stems from a business, management and human-resource context, and it describes a process to introduce and help a new employee become an effective member of their new organization by teaching them the knowledge, skills and behaviors required to succeed \cite{Bauer2011}. Graybill et al. state that onboarding is an important aspect to reduce the time for the employee to be a fully functional and engaged member of the new company \cite{GraybillJolieO;HudsonCarpenterMariaTaesil;OffordJeromeJr;PiorunMary;Shaffer2013}. Bradt et al. defines onboarding as "the process of acquiring, accommodating, assimilating, and accelerating new team members, whether they come from outside or inside the organization" \cite{Bradt2009}. Crumlish et al. \cite{Crumlish2009} writes that this definition and process apply well with user onboarding.

\subsection{User onboarding}
