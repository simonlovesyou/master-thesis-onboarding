\chapter{Background}
\label{chap:background}
This section introduce and explain key points and concepts which are essential to understand the context and application of this study.

\section{User Experience}
User-experience (UX) "...is about technology that fulfills more than just instrumental needs in a way that acknowledges its use as a subjective, situated, complex and dynamic encounter. UX is a consequence of a user’s internal state (predispositions, expectations, needs, motivation, mood, etc.), the characteristics of the designed system (e.g. complexity, purpose, usability, functionality, etc.) and the context (or the environment) within which the interaction occurs (e.g. organizational/social setting, meaningfulness of the activity, voluntariness of use, etc.)" \cite{Hassenzahl2006}

\subsection{Mobile UX}

\section{Onboarding}
The term "onboarding" is widely used in the context of organization socialization, and is described as the process where a new employee is introduced with the skills, knowledge and behaviors they need to succeed in their new organization \cite{Bauer2011}. In the field of management, recruiting and human resource management, onboarding is an important aspect to reduce the time for the employee to be fully functional and engaged with the new company \cite{GraybillJolieO;HudsonCarpenterMariaTaesil;OffordJeromeJr;PiorunMary;Shaffer2013}.

\subsection{User onboarding}
