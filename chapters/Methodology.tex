\chapter{Methodology}
\label{chap:methodology}
% In the study or research design, you explain where, when, how and with whom you are going to do the research.
% The question of 'how' will determine your research method. Are you gong to conduct research using a survey or perhaps with an experiment?

\section{Literature study}
A literature study is conducted to get an understanding of the research field of onboarding. The material for this study is mostly retrieved through the Google Scholar search engine and the "Digitala Vetenskapliga Arkivet" (DiVA) portal. This material is complemented with eventual articles, blogs and videos. The material gathered and the conclusion of the literature study is presented in the chapter \ref{chap:theoretical_framework}.

\section{Theoretical Background}
A theoretical background is developed from a body of material in the field of UX concerning different aspects which affect the users onboarding process.

\section{Best Practice Evaluation}
%
The best practice evaluation is done through a competitive review \cite{Schade2013} of other mobile applications to pinpoint the different kinds of methods used to onboard users.

\section{Framework}
A framework is developed based on the findings of the Theoretical background and best practice evaluation.

\section{User Testing}
%
The application is tested in a lab-setting with user who have not used the application before, but are experienced with other smartphone applications. \ignore{Find a source of user testing}

\section{Prototypes}
%
Prototypes are developed with the aid of the framework to try and fix eventual problems found when user testing. These prototypes are then evaluated through a heuristic usability evaluation to validate the developed framework.
