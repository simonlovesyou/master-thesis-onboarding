\chapter{Methodology}
\label{chap:methodology}
% In the study or research design, you explain where, when, how and with whom you are going to do the research.
% The question of 'how' will determine your research method. Are you gong to conduct research using a survey or perhaps with an experiment?

\section{Theoretical Framework}
The theoretical framework is built in two phases. The first phase is an extensive literature study and the second a collection of theoretical material

\begin{description}
  \item [Phase 1 - Literature study] The first phase consists of an extensive literature study of the research material of onboarding. The material gathered is gathered by searching the keywords "onboarding", "user onboarding" and "user sign up" on research platforms such as Google Scholar and "Digitala Vetenskapliga Arkivet" (DiVA) portal. This material is complemented with articles, blogs and videos of respectable individuals in the field.
  \item [Phase 2 - Theoretical Background] The material reviewed will lay grounds of what user experience (UX) is and what context it has to user onboarding. Some UX challenges and conventions related to mobile/smartphone devices is presented, and a vast coverage of some UX aspects important to onboarding such as \textit{Flow}, \textit{Perception}, \textit{Memory}, \textit{Attention}, \textit{Emotion and Motivation} and \textit{Learnability}.
\end{description}

The material gathered and the conclusion of the findings is presented in the chapter \ref{chap:theoretical_framework}.

\section{Best Practice Evaluation}
%
The best practice evaluation is done through a competitive review \cite{Schade2013} of other mobile applications to pinpoint the different kinds of methods used to onboard users.

\section{Framework}
A framework is developed based on the findings of the Theoretical background and best practice evaluation.

\section{User Testing}
The application is tested in a lab-setting with user who have not used the application before, but are experienced with other smartphone applications. \ignore{Find a source of user testing}

\section{Prototypes}
%
Prototypes are developed with the aid of the framework to try and fix eventual problems found when user testing. These prototypes are then evaluated through a heuristic usability evaluation to validate the developed framework.
lol
