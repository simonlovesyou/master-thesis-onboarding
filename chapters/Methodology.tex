\chapter{Methodology}
\label{chap:methodology}
% In the study or research design, you explain where, when, how and with whom you are going to do the research.
% The question of 'how' will determine your research method. Are you gong to conduct research using a survey or perhaps with an experiment?

The method of this study, as presented in the following sections, are based on (1) defining the framework and (2)
 evaluating the framework.
\section{Framework}
The framework we present is developed on top of the findings of the theoretical framework and the best practice evaluation.

\subsection{Theoretical Framework}
The theoretical framework is built in two phases: \textit{Theoretical background} and \textit{Best practice evaluation}.

\begin{description}
  \item [Phase 1 - Theoretical background] The reviewed material for the theoretical background lay grounds of what user experience (UX) is and what context it has to user onboarding. Some UX challenges and conventions related to mobile/smartphone devices is presented, and a vast coverage of some UX aspects important to onboarding such as \textit{Flow}, \textit{Perception}, \textit{Memory}, \textit{Attention}, \textit{Emotion and Motivation} and \textit{Learnability}.
  \item [Phase 2 - Best practice evaluation] The best practice evaluation is done through a competitive review \cite{Schade2013} of other popular mobile applications to pinpoint the different kinds of methods used to onboard users. The apps analyzed are chosen to cover a large ground of different onboarding techniques and apps of different categories (e.g game, productivity, social). Competitive analysis is important to generate insight into the competitors' employed strategies, both past and present, and will help us give an informed basis to develop and establish strategy advantages to our competitors \cite{Wilson2010}. The review will focus on what's relevant for this study and the topic of user onboarding.
\end{description}
% Coach marks
The material gathered and the conclusion of the findings is presented in the chapter \ref{chap:theoretical_framework}.

\section{Evaluation}
The evaluation and verification of the framework is performed in three steps, which are explained further in the following subsections. The first step (1) is by identifying possible onboarding pitfalls with the -app- through user testing, and (2) make amendments to the app with prototypes and the framework,  user test the prototypes and verify the improved user onboarding. These three steps are discussed further the

\subsection{User testing}

\subsection{Prototypes}

\section{User Testing}
The application is tested in a lab-setting with user who have not used the application before, but are experienced with other smartphone applications. \ignore{Find a source of user testing}

\section{Prototypes}
%
Prototypes are developed with the aid of the framework to try and fix eventual problems found when user testing. These prototypes are then evaluated through a heuristic usability evaluation to validate the developed framework.
