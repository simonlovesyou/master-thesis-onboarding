\chapter{Framework}
\label{chap:framework}

This chapter cover the design framework developed from the Theoretical background and best practice evaluation. In the following subsections we propose how each component of the framework contribute to a good onboarding experience. The framework is conceptualized around the organizational onboarding definition of Bradt. et al \cite{Bradt2009}, which constitute that onboarding is a process of \textit{acquiring}, \textit{acommodating}, \textit{assimilating} and \textit{accelerating} new members. Psychological, behavioral and UX research is cited to be able to facilitate a thorough understanding of how humans reason, learn and accomplish with mobile interfaces. With this framework, designers will be better equipped when designing onboarding experiences for mobile users.

\section{Acquisition}
There's a lot that determine the willingness for an individual to use and adopt some piece of technology. For this section we will present some aspects that affect the willingness.

\subsection{User Needs and Motivation}

The performance of a task is usually faciliated from a need. The need itself might be physiological, like eating to satisfy your hunger, or self-actualization like practicing painting because you want to be the best painter there is. When discussing needs, the \textit{Maslow's hierarchy of needs} is usually quoted.

\ignore{Maslow's hierarchy of needs}

To succesfully satisfy the need of your user with your app you need to identify the need they are trying to satisfy. The implicitly means that you have to identify you user-base and in what context they will be using your mobile app. In the product development cycle, this is usually done through \textit{user research}.

\subsection{Motivation}
Motivation is what drives human to accomplish things, keeps them interested and committed \ignore{\cite{Ryan and Deci, 2000}}.

Motivation can be categorized into \textit{intrinsic} and \textit{extrinsic} motivation. Intrinsically motivated people are motivated to do something due to the task being interesting or fun, like playing a video game, while extrinsically motivated people are motivated to do something to achieve a specific external goal, like taking a shower. Extrinsic motivation can be powerful, but when the external reward is removed so does the motivation. It has been found that intrinsically motivated people perform have a higher willingness to spend more time on a task. These categories are not mutually exclusice; a person may be both intrinsically and extrinsically motivated.

To improve the users intrinsic motivation, the user interface designer can relate the content and objectives of the application to the needs and interests of the learner \ignore{\cite{Keller1983}}. This can be done by using familiar metaphors and analogies \ignore{\cite{Curtis1984}}. Instructions provided to the user that use a personal style (e.g personal pronouns, names of specific people) rather than formal style may stimulate the user to learn. Also, providing immediate, positive and informative feedback in a context may improve intrinsic motivation, but not necessarily increase or decrease performance \ignore{\cite{(Bates, 1979; Condry, 1977; Deci, 1975, 1971; Keller, 1983)}}. Humor, on the the other hand, has been found to not improve motivation since it can distract the user and interfere with comprehension \ignore{\cite{Markiewicz, 1974; Sternthal & Craig, 1973}}.

Increasing the motivation of the user is not always the solution to make them perform a target behavior or use your mobile app. Sometimes increasing the users ability to perform the target behavior (making it easier to perform), or increasing the perceived ease of use, is the solution.

While some amount of motivation and ability is required for the user to performe some task, a trigger is required to remind the user to perform the task.

\subsection{User research}
To identify what

\section{Accommodation}

\section{Acceleration}

\section{Assimilation}
