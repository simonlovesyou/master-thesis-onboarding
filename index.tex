\documentclass[a4paper,11pt,twoside]{report}

%\usepackage{config/UmUThesis}           % Standard English
\usepackage[noindent]{config/UmUThesis}  % Non indented English
%\usepackage[se]{config/UmUThesis}       % Swedish

\usepackage[latin1]{inputenc}
\usepackage{courier}              % Nicer fonts are used. (not necessary)
\usepackage{pslatex}              % Also nicer fonts. (not necessary)
\usepackage{lmodern}              % Optional fonts. (not necessary)
\usepackage{url}
\usepackage[T1]{fontenc}

\usepackage{subcaption}
\usepackage{csquotes}

\usepackage{adjustbox}
\usepackage{graphicx}

\usepackage{blindtext}
\usepackage{amssymb}
\usepackage{pifont}

	\title{TITLE HERE}
	\author{}
	\supervisor{}
	\supervisore{}
	\examiner{Simon Johansson}
	\semester{Spring 2017}
	\course{Master Thesis, 30 hp}
	\education{M.Sc. Interaction Technology and Design, 300 hp}

\graphicspath{{resources/images/}}
\pagestyle{empty}

\begin{document}
\maketitle

%%-----------------------------------------------------------------------------------------------------------------
%%----------ACKNOWLEDGEMENTS---------------------------------------------------------------------
%%-----------------------------------------------------------------------------------------------------------------

\begin{center}
\section*{Acknowledgements}
Thanks mum and dad \ding{164}
\end{center}

\clearpage

%%-----------------------------------------------------------------------------------------------------------------
%%------------ABSTRACT-------------------------------------------------------------------------------------
%%-----------------------------------------------------------------------------------------------------------------

\begin{center}
\section*{Abstract}
Abstract, abstract, abstract...
% WRITE IN PRESENT TENSE!

%%% 1. Explanation of the title of your thesis
% The first function of the abstract is to further explain the title of your thesis. This allows readers of tour thesis to better determine if your thesis is interesting enough for them to read. A well-written abstract can encourage more people to consider your thesis important and, thus, to intend to read it.

%%% 2. Short version of your thesis
% The abstract serves as a short version for readers who don?t have the time to read the complete thesis. Often, managers and scientists read only the abstract and not the entire piece.

%%% 3. Overview of your thesis
% The abstract?s function is to serve as an overview of what readers can expect. This makes it easier for the reader to understand and to place in context the material in the thesis. A well-written abstract ensures that difficult material in your thesis is better understood.

% - What is the problem? Indicate the objective, problem statement and research questions of your thesis. If you have used hypotheses in your thesis, indicate them here.
% - What has been done? Briefly explain the method and approach of your research
% - What has been discovered? Provide a summary of the most important results and your conclusion
% - What do your findings mean? Summarize the key points from the discussion and present your recommendations

%% Understand your abstract without going through the rest of your thesis.
%% Include references when you use a source. You often do not use any references because you mainly write about your own findings and research

%% --------------------------- CHECKLIST -------------------------------------
% - Maximum of A4 sheet of paper
% - Placed after the preface and before the table of contents
% - Written in present tense
% - The objective is specified
% - The problem statement is given
% - The research questions or hypotheses are included
% - The methodology and approach of your research are briefly explained
% - A summary of the most important results are given
% - The conclusion is given (the answer to your research question / problem statement).
% - The results have been discussed and explained (discussion).
% - Suggestions for follow-up research are presented
% - Any recommendations are concisely discussed
% - The abstract clarifies what the thesis is about

\end{center}

\setcounter{page}{1}

%%-----------------------------------------------------------------------------------------------------------------
%%----------TABLE OF CONTENTS----------------------------------------------------------------------
%%-----------------------------------------------------------------------------------------------------------------

\tableofcontents

%% The chapter 'Introduction' will introduce the reader to field of research, mention some earlier and related work, introduce the company Doberman and the product which we will test
\chapter{Introduction}
\label{chap:introduction}

%% Establish the territory [the situation]
You only have one chance to make a good first impression. The amount of apps smartphone users download on a monthly basis has been steadily decreasing since the launch of the smartphone in 2004 (ref), and the founder of Mobilewalla suggest that users eventually uninstall 80-90\% of all downloaded apps \footnote{\url{http://usatoday30.usatoday.com/MONEY/usaedition/2012-01-31-App-Love-is-Fleeting\_ST\_U.htm}}. Similarly \cite{Perro2016} suggest that 75\% of apps get churned after 90 days, and 22\% of mobile users use an app only once before uninstalling \footnote{\url{http://www.slideshare.net/vaibhavkubadia75/mobile-web-vs-mobile-apps-27540693?from_search=1}}. Industry practitioners and researchers are increasingly mentioning the importance of app stickiness (retention) for a mobile applications success \cite{Perro2016} \cite{IGIGlobal2016} \cite{Kim2016} \footnote{\url{http://info.localytics.com/blog/an-open-letter-to-mary-meeker-we-are-in-a-mobile-engagement-crisis-localytics}}. And it is widely agreed across literature that stickiness will lead improved customer loyalty, positive word-of-mouth and revenue \cite{Reichheld2000} \cite{Srinivasan2002} \cite{Hsu2016a}.

The reason a user might uninstall an app and what makes an app "bad" has been previously researched \cite{Lin2012} \cite{Shklovski} \cite{Song2014}. \cite{IGIGlobal2016} has  The reasons found for uninstalling vary with privacy/security concerns, didn't meet users expectations, didn't use the app often etc. The term "app stickiness" is widely used in a marketing context for this type of problem to make sure that apps "sticks" to the users device. \cite{IGIGlobal2016} has linked user experience and its importance with app stickiness, and providing a good onboarding experience for the user will severely reduce abandonment rate \footnote{\url{https://clearbridgemobile.com/5-methods-for-increasing-app-engagement-user-retention/}} \footnote{\url{https://www.kahuna.com/blog/stop-spamming-your-users/}} and increase revenue \footnote{\url{http://www.appcues.com/blog/why-user-onboarding-is-the-most-important-part-of-the-customer-journey-by-2.6x/}}.

\ignore{Provide a definition of user onboarding}

In this paper, we present an analysis of psychology and user-experience on onboarding in mobile application. The ultimate aim is to provide guidelines for onboarding in different kinds of applications.

\section{Doberman}
Doberman is a digital design agency which help organizations and companies develop products and services that appeals customers. They have worked with companies like Sveriges Radio, Tele2 and Volvo Trucks. More information about Doberman can be found at their website\footnote{\url{http://doberman.co}}.

\section{Objective}
The study is set to develop a framework for user onboarding for usage with mobile apps, which is evaluated through user testing on mobile applications. The framework is meant to be general in regards to mobile applications, and its application support different kinds of mobile apps on different platforms.

\section{Limitations}
Any suggestions?

\section{Goals}
The goal is to produce a set of design guidelines which help the designer provide a good onboarding experience for a user in a mobile application.


%% The objective will state the problem, purpose & goal of the study and any restrictions.
\chapter{Objective}
\label{chap:objective}


%% The Methodology will present how the study will be conducted
\chapter{Methodology}
\label{chap:methodology}
% In the study or research design, you explain where, when, how and with whom you are going to do the research.
% The question of 'how' will determine your research method. Are you gong to conduct research using a survey or perhaps with an experiment?

The method of this study, as presented in the following sections, are based on \begin{enumerate*}[label=(\(\arabic*\))]
  \item creating a basis for the framework through a theoretical framework (see Section \ref{sec:theoretical_framework}, below); and
  \item evaluating the framework (see Section \ref{sec:evaluation}, below)
\end{enumerate*}

\section{Theoretical Framework}
\label{sec:theoretical_framework}
The theoretical framework is built in two phases: \textit{Theoretical background} and \textit{Best practice evaluation}.

\begin{description}
  \item [Theoretical background] Relevant theoretical material is gathered to explain the onboarding process defined by Bradth et al. \cite{Bradt2009} in a user experience research context, with the additional material about relevant UX challenges. The material gathered is retrieved through the Google Scholar search engine and the "Digitala Vetenskapliga Arkivet" (DiVA) portal. A review of the material gathered is presented in Section \ref{sec:theoretical_background}.
  \item [Best practice evaluation] Common onboarding techniques and their benefits and disadvantages are analyzed. A review of the analyses and the conclusions of it is presented in Section \ref{sec:best_practice_evaluation}.
\end{description}

The material gathered and the conclusion of the findings is presented in the chapter \ref{chap:theoretical_framework}.

\section{Evaluation}
\label{sec:evaluation}
The evaluation and verification of the framework are performed in three phases, which are explained further in the following subsections. The first phase \begin{enumerate*}[label=(\(\arabic*\))]
  \item is by analyzing the user interface of the conceptual mobileapp design with the aid of the framework, the second phase
  \item \label{enum:prototype}is making changes to the design with the framework in an interactive prototype, and the last phase
  \item \label{enum:usertest}is user test and analyze the onboarding process, make amendments to the framework as necessary
\end{enumerate*}

It is worth to note that when the conceptual mobileapp design is being analyzed it is done with no consideration of any planned functionality by company, and is only analyzed from a user onboarding perspective provided by the framework.

Phase \ref{enum:prototype} and \ref{enum:usertest} is further explained in Section \ref{subsec:prototypes} below.

\section{Prototypes}
\label{subsec:prototypes}
Design prototypes are developed with the aid of the framework. Prototypes are a working representation built to test design ideas, and are often tested by observing users carry out intended tasks with the interface \cite{Walker2002}.

\subsection{Usability Testing}
The usability tests are conducted in a lab setting, where the participants are asked to perform tasks with the developed prototype of the mobileapp app.

The tests are planned after these specific steps, as defined by \cite{Dumas1999}:

\begin{itemize}[noitemsep]
  \item defining the goals and concerns that are driving the test
  \item deciding who should be participants
  \item recruiting participants
  \item selecting and organizing tasks to test
  \item creating task scenarios
  \item deciding how to measure usability
  \item preparing other materials for the test
  \item preparing the testing environment
  \item preparing the test team---assigning specific roles, training team members, and practicing before the test starts
  \item conducting a pilot test and making changes as needed
\end{itemize}

The steps have been categorized into \textit{goals}, \textit{participants}, \textit{tasks}, \textit{scenarios}, \textit{user tests} and \textit{measurements}, all of which are explained further in below.

\subsubsection{Goals}
\label{subsubsec:goals}
The goals of the usability tests of mobileapp we identify are \begin{enumerate*}[label=(\(\arabic*\))]
  \item a general goal for the product,
  \item quantitative usability goals for the product,
  \item general concern for this test,
  \item specific concerns for this test
\end{enumerate*}, as proposed by \cite{Dumas1999}.

\subsubsection{Participants}
\label{subsubsec:participants}
The participants are chosen from the target market of mobileapp app, which have no prior usage of the mobileapp app. This decision is made to more easily identify learnability issues with the interface as they use it for the first time. The identified characteristics of the test group follow the user profile template suggested by \cite{Dumas1999}:

\begin{enumerate}
  \item General characterization of the user population
  \item \label{enum:characteristics}Characteristics of the users that are relevant to the test
  \item Which of the characteristics that are listed in \ref{enum:characteristics} should all users in the test have in common and how will you define them?
  \item Which of the characteristics that you listed in \ref{enum:characteristics} will vary in the test and how will you define them?
\end{enumerate}

The profile of the participants are represented through \textit{personas} based on a survey about their habits and secondary internal information from past user research done by Doberman in collaboration with company. The final profile presented as Personas can be found in Section \ref{sec:participants}.

\subsubsection{Tasks}
\label{subsubsec:tasks}

Since we cannot test every possible task users can do with the product, we have to carefully select the task they are to perform. Dumas and Redish \cite{Dumas1999} write that we should select a sample of tasks
\begin{itemize}[noitemsep]
  \item ... that probe potential usability problems
  \item ... suggested from your concern and experience
  \item ... derived from other criteria
  \item ... that users will do with the product
\end{itemize}

For each identified task which is defined with the help of the developers and product owner, as suggested by \cite{Dumas1999}, we identify the time required for the task and resources needed. All tasks defined are detailed in a final list of tasks which can be found in Chapter \ref{chap:result}.

\subsubsection{Scenarios}
The tasks identified in section \ref{subsubsec:tasks} are formulated to the test participants as \textit{scenarios}. Scenarios describe the task in the user's --- not the product's --- words in an unambiguous way. A good scenario is directly linked to your tasks and concern and gives the user's enough information to complete the task. The participants where asked to think out loud when performing the scenarios, and depending on the participants ability to do so the test leader might probe the user with questions about how they think and why/how they performed an action.

\subsubsection{User test}
The user tests are prepared before hand with a script for the test supervisor, a consent form, test location and a post-test questionnaire for the participants. The script contains the following items for each scenario:

\begin{description}
  \item[Concern] The concern that is being examined for this particular scenario, formulated as a question.
  \item[Test setup] A general description of the setup that is given to the test participant.
  \item[Task description] A short description of the tasks the user has to carry out to complete the scenario.
  \item[Scenario] The scenario description that is verbally given to the test participant.
  \item[Post-scenario questions] Probing questions to ask the test participant of what they thought about the scenario.
\end{description}

A pilot test is carried out in order to identify flaws with the test setup, to practice the test procedure and to identify critical bugs in the prototype before the actual tests are carried out.

\subsubsection{Measurements}
\label{subsubsec:measurements}
Dumas \cite{Dumas1999} suggest that the performance measures that we select should be directly related to the concerns and usability goals that we define for this particular usability test, as \cite{Dumas1999} suggest.

The performance measures for the usability test includes the subjective opinions of the test participants --- people's perceptions, opinions and judgments. They were documented by the test supervisor during the scenarios.

For collecting quantitative measures, we measure the perceived ease of use and perceived usefulness pre- and post-test of the application, inspired by the work of Davis\cite{Davis1989} and his research into the technology acceptance model.


%% The literature review will present the relevant literature in the field of onboarding
\chapter{Study of User Onboarding}

% A method used to gather knowledge that already exists in relation to a particular topic or problem.
% Conducting literature research provides insight into existing knowledge and theories related to your topic. It also ensures that your thesis has a string scientific grounding.
% Your goal is instead to critically discuss the most relevant ideas and information that you have found as part of your theoretical framework.

% The literature review serves as a real cornerstone for analyzing the problem being investigated. The basis for developing a comprehensive theoretical framework.

% Once you have a general idea of the problem and research questions you would like to address in your thesis, the first step is often to begin reviewing the literature.
% The insights into the existing knowledge and theories that you will gain through the literature review will also help you to establish a strong scientific starting point for the rest of your research.

%%%% LITERARURE ROADMAP
%% 1. Prepare
% The first step involves orienting yourself to the subject in order to get a more global picture of the area of inquiry. It also entails making a list of keywords, which will serve as the basis for the next step.

%% 2. Collect literature
% Does the same authors name keep popping up? This usually means this individual has done lot of research on the topic.

%% 3. Evaluate and select the literature
% Start with just the introduction and conclusion

%% 4. Process the literature
% Ask yourself the following questions while reading:
% - What is the problem being investigated and how is the research addressing it?
% - What are the key concepts and how are they being defined?
% - What theories and models does the author use?
% - What are the results and conclusions of the study?
% - How does this publication compare to related publications in this field?
% - How can I apply this research to my own research?


%% The theoretical framework will present material for us to understand the resulting framework, which is also presented in this chapter, which also contain the results from the best practice analysis.
%% The basic red thread of this theoretical framework

%% Problems
% Problems UX designers have to consider (UX is defined elsewhere)
% Problems specific to mobile - no long sessions etc, small attention span, app stickiness etc
%% Usage
% Motivation - What drives the user to use a app?
% App discoverability - Mention the problem, but do not go into depth
% What are the kinds of behavior that can be used when using the application? (skill, rule, knowledge) - Segment into mental models
% Mental models - What are they and how do they work? How are they relevant to UX? How are they measured? - Segment into Gulf of evaluation and execution
% Gulf of evaluation and execution - Description and relevance
%% Bridging the gap
% Direct manipulation
% Learnability - Could be seen as adjusting the mental model in this context?
% Feedback and response through Human Computer Dialogue?
%% Design
% Cognitive load - How to design to reduce cognitive load
% Affordances / Signals
% Incidental learning
% Memory
% Language - Articulatory and Semantic distance

\chapter{Theoretical Framework}
\label{chap:theoretical_framework}
The theoretical framework is performed in two phases. The first phase \textit{Literature review} establish the current research environment in regards to onboarding, which lay ground for this study. The purpose of the second phase is to develop the \textit{theoretical background}, which major parts of the framework is based on. The theoretical background is then further supported by best practice evaluations of other mobile applications.

\begin{description}
  \item [Phase 1] The material is for the first phase is retrieved by searching the keywords "onboarding", "user onboarding" and "user sign up".
  \item [Phase 2] The material reviewed will lay grounds of what user experience (UX) is and what context it has to user onboarding. Some UX challenges and conventions related to mobile/smartphone devices is presented, and a vast coverage of some UX aspects important to onboarding such as "Flow", "Perception", "Memory", "Attention", "Emotion and Motivation" and "Learnability".
\end{description}

\section{Literature study}
% Literature reviews are designed to provide an overview of sources you have explored while researching a particular topic and **to demonstrate to your readers how your research fits into the larger field of study**
\subsection{User Onboarding}
There hasn't been a lot of academic research in the field of user onboarding, even though there's a lot of research regarding employee and manager onboarding (ref) (ref) (ref). The most notable contributions to the research field of user onboarding is by Oechsner \cite{Oechsner2016}, where she presents general guidelines for onboarding a user to a digital platform. The guidelines she presents are supported by psychology and IxD research, but fail to include the body of work of employee onboarding and the marketing aspect \textit{stickiness}. Also, her guidelines are meant to be general and may not alone aid the entire onboarding process of all platforms. Her study of user onboarding has been exhaustive, but with the circumstances related to the different usage model of mobile applications support the need for revisiting onboarding process.

\subsection{Onboarding}
\ignore{Include more onboarding sources}

\section{Theoretical Background}
% User problems -> Mobile -> Value proposition
To better understand how to provide a good onboarding experience for the mobile app user we have to understand the problems associated with UX-design, both in general and problem mobile devices. After defining these problems we draw inspiration from cognitive and behavioral psychology. Cognitive psychology is a branch of psychology which study higher mental processes of the human mind such as attention, language use, memory, perception, problem solving, and thinking \footnote{\url{http://www.apa.org/research/action/glossary.aspx?tab=3}}. Behavioral psychology and its research field is focused on the environmental determinants of learning and behavior.

%% PROBLEMS
\subsection{UX Problems}
The field of UX is concerned with problems such as ... \ignore{find source that present these problems}

\subsection{Mobile devices}
Designing for mobile devices such as smartphones introduce a number of different unique problems such as small screen sizes, limited connectivity, limited battery and restricted set of available inputs \cite{Zhang2005}. One of the biggest challenges is to consider the context of which they are used in \cite{Zhang2005} \cite{Harrison2013} \cite{Korhonen2010}. \cite{Korhonen2010} has identified eight possible contextual factors one has to consider when designing for mobile devices;
\begin{description}
  \item [Environment Context] The environment context describe the surrounding area of the user and the other entities it contains which can affect the user directly or indirectly.
  \item [Personal Context] The personal context describe both the physiological characteristics and attributes (pulse, blood pressure and hair color etc), and the mental attributes of the user (mood, expertise and stress). Mental attributes are most often considered when designing for the context of use.
  \item[Task Context] The task context describe the events, actions and activities the user is currently engaged in. This context also describe if the use of a device is a primary or secondary task.
  \item[Social Context] Expand
  \item[Spatio-Temporal Context] Expand
  \item[Device Context] Expand
  \item[Access Network Context] Expand
\end{description}
While modern smart phones are packed with a lot of functionality (GPS navigation, voice and data communication, multimedia consumption, gaming) their small form factor limit the possible input and outputs. The two primary means of input supported by these kind of devices are
\begin{enumerate*}
  \item Touchscreen and
  \item Sensors
\end{enumerate*}

The sensors available are dependent on the device, but mostly encumbers accelerometers, gyroscopes and orientation sensors. The touchscreen enables the user to interact through two-dimensional surface gestures \cite{Ruiz2011}. It is also possible for the user to gesture in three dimensions using the sensors, but we will focus mostly on twodimensional gestures. The possible actions and models which we interact through with mobile devices are mostly based on hand-based gestures. The gesture model which we have for touchscreen-based mobile devices include tapping, swiping, shaking,

... Discuss different app categories and their different properties

\subsection{Motivation}
To understand the users motivation to use an app we have to understand the needs of the user, and the different kinds of value important to the user. Maslow’s theory of motivation \cite{Maslow1943} say that humans are motivated by a hierarchy of needs. He explains that when one hierarchy of need has been satisfied the human try to satisfy the next need in the hierarchy, and that the subsequent need has to be fulfilled to be able to care for any of the other needs; "A person who is lacking food, safety, love, and esteem would most probably hunger for food more strongly than for anything else." To be able to understand how to provide a good onboarding experience for the user we have to understand the motivation behind the user to use an app. To do so we have to consider the users’ goals and their present condition; their need. Their need is

\subsection{Discoverability}
App discoverability is the concern of finding the correct app. \ignore{Expand, a lot}

\subsection{Behavior}
\cite{Rasmussen1983} has identified three typical levels of performance. \textit{The skill-based behavior} represent the subconscious actions and activities which we perform on a "smooth, automated, and highly integrated patterns of behavior". This level of performance is based on a simple feedback loop, where a stimuli facilitate a motor output. \ignore{Provide example of this}

\textit{Rule-based behavior} are patterns of behavior which emerge from previous and similar actions performed in a previous similar occasion. The user acts in a goal-oriented manor where the user tries to achieve a goal, which is usually not explicitly stated, with the rules that has empirically evolved through previous successful experiences. \ignore{Find evidence that rule and skill-based behavior might help the user with cognitive load}

The main differences of skill-based and rule-based behavior is that the person may not consciously be aware of the actions the user perform when performing actions on a skill-based level, and may not be able to recollect why such an action has been taken. The higher level rule-based actions are generally based on "explicit know-how, and the rules used can be reported by the person".

The third, and final level of performance as \cite{Rasmussen1983} has identified, is \textit{knowledge-based}. Knowledge-based performance is used during situations that the user is not familiar with, and where any know-how or rules cannot be used from previous experiences. \cite{Rasmussen1983} states that "In this situation, the goal is explicitly formulated, based on an analysis of the environment and the overall aims of the person. Then a useful plan is developed-by selection-such that different plans are considered, and their effect tested against the goal, physically by trial and error, or conceptually by means of understand the functional properties of the environment and prediction of the effects of the plan is considered." At this level of abstraction, the user represent the system in an internal construct called \textit{mental model}.

%% Cognitive models -> Mental models ->

\subsection{Mental model}
A common concept in the field of cognitive psychology is the concept of \textit{mental models}. Mental models was first introduced by American philosopher Charles Sanders Peirce, where he argues that humans reason by a process which
"...examines the state of things
asserted in the premises, forms a diagram of that state of things, perceives in the parts of that diagram relations not explicitly mentioned in the premises, satisfies itself by mental experiments upon the diagram that these relations would always subsist, or at least would do so in a certain proportion of cases, and concludes their necessary, or probable, truth." \cite{Pierce1974}

This was further elaborated by the psychologist Kenneth Craik, where he proposes that humans construct "small-scale models" of external reality \cite{Craik1967}. These mental models enable us to use past events to be able to react to present events and anticipate future events, to reason, and to understand our environment. Since Craik's contributions, cognitive psychologist has argued that mental models are formed through current and general knowledge, perception and imagination \cite{Johnson-Laird2001} \ignore{Find some more citations to support this claim}. In the field of Human-Computer Interaction (HCI) mental models has sometimes been used interchangeably with cognitive and conceptual models, and their difference and usage might be confusing as \cite{Staggers1993} concludes. For the purpose of this paper, we'll be using Normans \cite{Norman2013a} definition of conceptual and mental model, where he states that mental model needs to be consistent with the designers conceptual model. Mental models are the models made from experience, instruction and training which users interact through \cite{Norman2013a}. Mental models of devices are created mostly through perceiving possible actions and its visible structures afforded by its interface \cite{Norman2013a}. The mental model of the user guide the users expectation of the application, and can guide the users navigation, planning of actions and contribute to the interpretation of interfaces feedback \cite{Jin1992}\ignore{not the actual source}. When the user has acquired an adequate mental model of the structure and possible functions of the app the user is less likely to become disoriented \cite{Jih1992}.

Norman \cite{Norman2013a} provide us with a seven-stage model which describe the manor by which users use interactive products.
\begin{enumerate}
  \item Forming the goal
  \item Forming the intention
  \item Specifying the action
  \item Executing the action
  \item Perceiving the system state
  \item Interpreting the system state
  \item Evaluating the outcome
\end{enumerate}

This model of actions help us describe how an user explore an interface \cite{}\ignore{Polson and Lewis, 1990}. This approximate model \cite{Norman2013a} is not consciously used by the user, but rather it tries to explain how we perform tasks. The model is cyclic, meaning that the user will experience multiple loops of the model as they explore an interface. The model is developed on the basis that human actions has two aspects; execution and evaluation. Execution is doing something which affect the world, and evaluation is the comparison if the world reached or got closer to a state which was desired by the user. These two aspects have their of stages of performance. The stages of execution involves forming the goal, which may be an abstract representation of what we want to achieve (get to work, eat dinner, pick a movie). The goal forms our intentions to perform an action, which constitutes in an action sequence to be executed to satisfy the intention. Finally in the stage of execution, the action sequence is executed upon the world and we reach the stages of evaluation. The first stage of evaluation is perceiving the systems state. The perception is then interpreted according to our expectations and finally compared to our intentions and goal. As the users try to finish their tasks, there are four critical points where user errors can occur, as identified by \cite{Shneiderman2004}:
\begin{enumerate*}
  \item users form an inadequate goal,
  \item users might not find the correct interface touchpoint because of a label or icon that does not sufficiently represent its corresponding action
  \item users may not know what action to perform to get a desired output and
  \item users get misleading feedback from the system
\end{enumerate*}

%%BRIDGING THE GAP

\subsection{Gulf of evaluation and execution}
% Expand this section.
The user initially start with an intention of achieving a goal, where the goal is often expressed in psychological terms. The system or interface

\subsection{Direct manipulation}

Direct manipulation was first introduced by Shneiderman, which describe a system which inherit the following properties (p. 184 erryday things as well):

\begin{enumerate}
  \item Continuous representation of the object of interest
  \item Physical actions or labeled button presses instead of complex syntax
  \item Rapid incremental reversible operations whose impact on the object of interest is immediately visible
\end{enumerate}

\subsection{Learnability}
More often than not, "interface usage requires learning" \cite{Grossman2009}. Even though there’s plentiful of research regarding learnability, its definition is not widely agreed upon. The different definitions consider different scopes of learnability, e.g. Nielsen definition consider the initial learning curve of the user, and that a highly learnable system would be "allowing users to reach a reasonable level of usage proficiency within a short time." What "a reasonably level of usage proficiency" is relative to a "short time" is still unclear, and it leaves a lot to our own imagination. Schneiderman et al. provide us with a more general applicable definition of "the time it takes members of the user community to learn how to use the commands relevant to a set of tasks". Schneiderman et al. also discuss a different aspect which tightly coupled with learnability, which is \textit{Retention over time}; How well is the user able to recall how to use an application after a specific time period? Retention may

\subsubsection{Feedback}
Feedback...

\subsubsection{Human-Computer Dialogue}

If the conceptual model is not clearly communicated to the user and is not properly corrected through computer-human dialog, the user will have trouble performing the tasks they've set out to solve their problem. Studies of blame \ignore{Find these studies}, or \textit{attribution}, has shown that when a fault occurs in a system, the person in question is more likely to attribute the error to system than their own. Similarly, when fortune occurs to a person they are likely to attribute the fortune to their person and intelligence.

If the users mental model is not consistent with the conceptual model provided by the designer, the users mental model can be modified through a \textit{computer-human dialog}

The communication between a user and a computer-based system is through a \textit{user-computer dialogue} \cite{Foley1996}, where the dialogue is communicated through a language of inputs and outputs. Similarly to regular human-to-human conversation, we may provide feedback if something is misunderstood or help the other person finish a sentence. The same is true for human-to-computer communication, where feedback is used to reinforce or discourage the users action, making the user adjust his or her mental model of the system.

The area of psychology that focuses on the environmental determinants of learning and behavior.

%% DESIGN
\subsection{Cognitive load}
The larger the amount of available information is given to the user from an interface, the more likely it is for the user to fall under the pressure of excessive cognitive load. According to \cite{Jih1992}, the user of an interface has to endure three different types of cognitive load; the content of the application, the application structure and the responses and feedback given by the interface. Schneiderman et al. state that "Providing excessive functionality ... is also a danger, because the clutter and complexity make implementation, learning, and usage more difficult".

\subsection{Affordances / Signals}

\subsection{Incidental and Informal learning}
Incidental learning is unintentional or unplanned learning that results from performing activities \cite{Kerka2000}; activities which primary objective is not to acquire knowledge but to progress while pursuing a goal. As Kerka \cite{Kerka2000} has identified in her literature review of incidental learning, incidental learning may occur "through observations, repetition, social interaction, and problem solving; ...; from mistakes, assumptions, beliefs, and attributions; or from being forced to accept or adapt to situations". Incidental learning at this time was discussed in the context of workplace learning, but as \cite{Marsick2001} conclude, this type of learning happen through "everyday encounters while working and living in a given context". Jones et al. \cite{Jones2014} has drawn the conclusion that this type of learning is particularly suited for smartphone use.

\subsection{Memory}
\subsection{Language}
\subsection{Interface design}
To be able to effectively use the intrinsic motivation of the user, it is


%% The result chapter will contain the results from the expert interviews, any eventual framework remarks, user tests, personas and prototypes.
\chapter{Result}
\label{chap:result}
% Carry out the research design that you described in the previous chapter.
% Describe how the research went and analyze the result.

\section{Literature study}
%
\section{Interviews}
%
\section{Best Practice Evaluation}
%
\section{Application review}
%
\section{User Testing}
%
\section{Personas}
%
\section{Prototypes}
%


%% The discussion chapter will discuss the results found of the study and its implications.
\chapter{Discussion}
\label{chap:discussion}
% WRITE IN PRESENT TENSE

% In the discussion, you write mor interpretatively and colorfully about the results. Whereas you kept it concise in the conclusion, you write more in-depth about the subject in the discussion section.
% Evaluate the research: you may discuss your expectations of possible causes of and consequences of the results, possible limitations and suggestions for follow-up research.

% Start your discussion with the validity of your research. Then discuss the results and indicate whether they meet your expectations.
% In this section, you will give explanations for meeting or not meeting these expectations.
% Describe how your results fit with the framework that you have drawn in the first chapter (introduction, motivation, theoretical framework, and research questions/hypotheses).
%Show how the finding provide new or different insights into what was already known.

% Present the limitation of your research in a new paragraph within the discussion. Describe which observations you can make based on the research results.
% If there are some side notes that can be made to the research or you were hindered by certain limitations, these issues can explain of the results you obtained.
% Name these, but also explain how there factors can be improved in the future research.

%% --------------------------- CHECKLIST -------------------------------------
% - The validity of the research is demonstrated.
% - New insights are explained.
% - The limitations of the research are discussed.
% - It is indicated whether expectations were justified.
% - Possible causes and consequences of the results are discussed.
% - Suggestions for possible follow-up research are made.
% - Own interpretations have been included in the discussion.
% - There are no suggestions for follow-up research that are too vague.


%% The chapter Conclusion will conclude the study.
\chapter{Conclusion}
% PRESENT TENSE
% Should be between 200-400 words.

% Answer your research question. Reiterate the research question, but integrate an explanation of it into the rest of the sections discussion.

%% --------------------------- CHECKLIST -------------------------------------
% - The research questions have been answered.
% - The main question or problem statement has been answered.
% - The hypotheses have been confirmed or refused.
% - The right verb tense has been used.
% - No issues are interpreted.
% - No new information has been given.
% - No examples are used.
% - No extraneous information is provided.
% - No passages from the results have been cut and pasted.
% - The first person has not been used.




%%-----------------------------------------------------------------------------------------------------------------
%%----------------APPENDIX----------------------------------------------------------------------------------
%%-----------------------------------------------------------------------------------------------------------------

\clearpage

\addcontentsline{toc}{chapter}{\bibname}
\bibliographystyle{unsrt}

\bibliography{library}

\appendix

\end{document}
